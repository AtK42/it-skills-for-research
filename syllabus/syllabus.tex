\documentclass[12pt]{article}

\usepackage[margin=1in]{geometry}

% No indent for new paragraphs
\setlength\parindent{0pt}
\setlength\parskip{12pt}

\title{IT skills for research \\[24pt] \large syllabus}
\author{
    Dmitry Borisenko 
    \and
    Igor Pozdeev
}
\date{}

% ---------------------------------------------------------------------------

\begin{document}

\maketitle

\section*{introduction} % (fold)
\label{sec:introduction}

This course's sole purpose is to introduce an aspiring researcher to certain useful software and IT concepts which are to research what icing is to cake: not the defining part but not inessential either. After all, research is not only idea generation, data analysis and publication -- there are elements such as project and data management, collaboration, knowledge transfer etc., and while universities do a good job at teaching the former, the latter are often underrepresented in curricula, despite being efficiency boosters and impact multipliers.

Many a concept to be taught in this course are no big news in software development, a field in lots of ways similar to academic research: a developer starts with an idea, sets up a team, writes code, runs tests, corrects errors, creates documentation, ensures user friendliness and publishes the product -- to any of these a scholar would find an analogue in the own routine. The similarities ensure that knowledge transfer is possible, and since the IT is the more dynamic field, the scholars mostly end up at the receiving end. Good for us -- we can learn!

% section introduction (end)

\newpage
\section*{objectives} % (fold)
\label{sec:objectives}

Learn some of the best practices required to efficiently conduct research today:

\begin{itemize}
    \item command line:
    \item[] use cases, commands, scripts, environment variables, file manipulations, remote connection, authentication, cronjob;
    \item project management:
    \item[] development environment, directory layout, text editors, coding modes;
    \item version control (git):
    \item[] use cases, installation, git areas, tracking files, undoing things, concurrent editing, branching, github;
    \item latex:
    \item[] installation, formats and engines, command line interface, packages, beamer, custom styles, bibliographies;
    \item data management:
    \item[] organizing a database, database api, memoization, data file types, web api, introduction to SQL;
    \item visualization:
    \item[] cognition science basics, colors, tricks for better visuals;
    \item knowledge transfer:
    \item[] interactive apps, jupyter notebooks;
    \item reproducibility:
    \item[] open source concepts, environment export, makefiles, testing.
\end{itemize}

% section objectives (end)

\end{document}