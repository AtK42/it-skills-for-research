\documentclass[12pt]{article}

\usepackage[left=1in, right=1in, top=0.75in, bottom=0.75in]{geometry}

\usepackage{enumitem}
\usepackage{tabularx}
\usepackage[dvipsnames]{xcolor}
\usepackage[hidelinks]{hyperref}
\hypersetup{
    colorlinks,
    linkcolor=NavyBlue
}

% for spacing in ToC
\usepackage{tocloft}
\renewcommand\cftsecafterpnum{\vskip15pt}

% No indent for new paragraphs
\setlength\parindent{0pt}
\setlength\parskip{12pt}

\title{IT skills for research \\[24pt] \large syllabus}
\author{
    Dmitry Borisenko 
    \and
    Igor Pozdeev
}
\date{}

% ---------------------------------------------------------------------------

\begin{document}

\maketitle

\tableofcontents

\newpage

\section{introduction} % (fold)
\label{sec:introduction}

This course's sole purpose is to introduce an aspiring researcher to certain useful software and IT concepts which are to research what icing is to cake: not the defining part but not inessential either. After all, research is not only idea generation, data analysis and publication -- there are elements such as project and data management, collaboration, knowledge transfer etc., and while universities do a good job at teaching the former, the latter are often underrepresented in curricula, despite being efficiency boosters and impact multipliers.

Many a concept to be taught in this course are no big news in software development, a field in lots of ways similar to academic research: a developer starts with an idea, sets up a team, writes code, runs tests, corrects errors, creates documentation, ensures user friendliness and publishes the product -- to any of these a scholar would find an analogue in the own routine. The similarities ensure that knowledge transfer is possible, and since the IT is the more dynamic field, the scholars mostly end up at the receiving end. Good for us -- we can learn!

% section introduction (end)

\section{objectives} % (fold)
\label{sec:objectives}

Learn some of the best practices required to efficiently conduct research today:

\begin{itemize}[topsep=0pt, itemsep=4pt, partopsep=0pt, parsep=0pt]
    \item \textbf{project management}:\par
        development environment, directory layout, text editors, coding modes;
    \item \textbf{command line}:\par
        use cases, commands, scripts, environment variables, file manipulations, remote connection, authentication, cronjob;
    \item \textbf{version control (git)}:\par
        use cases, installation, git areas, tracking and ignoring files, undoing things, concurrent editing, branching, github;
    \item \textbf{latex}:\par
        installation, formats and engines, command line interface, packages, beamer, custom styles, bibliographies;
    \item \textbf{data management}:\par
        organizing a database, database api, memoization, data file types, web api, introduction to SQL;
    \item \textbf{visualization}:\par
        cognition science basics, colors, tricks for better visuals;
    \item \textbf{knowledge transfer}:\par
        interactive apps, jupyter notebooks;
    \item \textbf{cloud computing}:\par
        remote connections, virtual machines, Amazon Web Services (AWS);
    \item \textbf{reproducibility}:\par
        open source concepts, environment export, makefiles, testing.
\end{itemize}

% section objectives (end)

\section{requirements} % (fold)
\label{sec:requirements}

Intermediate knowledge of Python or R: how to write a function, plot a chart, format string values, read to and write from text files etc.; previous exposure to research projects.

% section requirements (end)

\section{schedule} % (fold)
\label{sec:schedule}

\begin{tabularx}{0.7\textwidth}{Xl}
    week 1 & introduction + software + project management \\
    week 2 & command line \\
    week 3 & version control p1: working solo \\
    week 4 & version control p2: collaboration \\
    week 5 & latex \\
    week 6 & data management p1: setting up a database \\
    week 7 & data management p2: web api \\
    week 8 & data management p3: introduction to SQL \\
    week 9 & visualization \\
    week 10 & knowledge transfer \\
    week 11 & reproducibility \\
    week 12 & cloud computing
\end{tabularx}

% section schedule (end)

\section{software} % (fold)
\label{sec:software}
We will be using Slack, a command line terminal, Python or R plus jupyter (potentially with an IDE), an advanced text editor (such as Sublime Text, Atom or VS Code), SQL management software, git (potentially with dedicated software or integration into the text editor of choice), as well as LaTeX (potentially with dedicated editors).

% section software (end)

\newpage
\section{examination} % (fold)
\label{sec:examination}

Please work in groups of 2-4 people and set up one repository per group -- this will be the place to keep the midterm and final assignments on separate branches. Groups must be formed before the 4th lecture starts, the only allowed change after this deadline being leaving one.

\subsection{midterm} % (fold)

\textbf{(25\%, group assignment, same grade)}

Please do the first 6 end-of-chapter exercise sets by the beginning of the 7th lecture and push the solutions to branch ``midterm''.

\label{sub:midterm}

% subsection midterm (end)

\subsection{final} % (fold)
\label{sub:final}

\textbf{(75\%, group assignment, same grade)}

Please submit a little research project adhering to the following requirements:

\begin{itemize}[topsep=0pt, itemsep=4pt, partopsep=0pt, parsep=0pt]
    \item make it hosted as a single repository on Github with all group members as contributors;
    \item structure it neatly and concisely, avoid the clutter of folders and files for which third people have no use (use .gitignore);
    \item keep working on it consistently, avoid bulk commits;
    \item create a little well-documented database with the data you use, featuring possibility to update it with fresh values;
    \item design figures and tables to support your findings, keeping them in line with visualization standards discussed in class;;
    \item write a short paper about it using LaTeX, populating it with sections, a table of contents, as well as the above tables and figures;
    \item write a beamer presentation for your project using LaTeX;
    \item create one interactive app (R shiny or jupyter notebook) describing the main finding and presenting several robustness checks;
    \item make the results reproducible by ensuring that the coding environment is exported and concise documentation is present.
\end{itemize}

% subsection final (end)

% section examination (end)

\end{document}